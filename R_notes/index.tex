% Options for packages loaded elsewhere
\PassOptionsToPackage{unicode}{hyperref}
\PassOptionsToPackage{hyphens}{url}
\PassOptionsToPackage{dvipsnames,svgnames,x11names}{xcolor}
%
\documentclass[
  letterpaper,
  DIV=11,
  numbers=noendperiod]{scrreprt}

\usepackage{amsmath,amssymb}
\usepackage{iftex}
\ifPDFTeX
  \usepackage[T1]{fontenc}
  \usepackage[utf8]{inputenc}
  \usepackage{textcomp} % provide euro and other symbols
\else % if luatex or xetex
  \usepackage{unicode-math}
  \defaultfontfeatures{Scale=MatchLowercase}
  \defaultfontfeatures[\rmfamily]{Ligatures=TeX,Scale=1}
\fi
\usepackage{lmodern}
\ifPDFTeX\else  
    % xetex/luatex font selection
\fi
% Use upquote if available, for straight quotes in verbatim environments
\IfFileExists{upquote.sty}{\usepackage{upquote}}{}
\IfFileExists{microtype.sty}{% use microtype if available
  \usepackage[]{microtype}
  \UseMicrotypeSet[protrusion]{basicmath} % disable protrusion for tt fonts
}{}
\makeatletter
\@ifundefined{KOMAClassName}{% if non-KOMA class
  \IfFileExists{parskip.sty}{%
    \usepackage{parskip}
  }{% else
    \setlength{\parindent}{0pt}
    \setlength{\parskip}{6pt plus 2pt minus 1pt}}
}{% if KOMA class
  \KOMAoptions{parskip=half}}
\makeatother
\usepackage{xcolor}
\usepackage{soul}
\setlength{\emergencystretch}{3em} % prevent overfull lines
\setcounter{secnumdepth}{5}
% Make \paragraph and \subparagraph free-standing
\ifx\paragraph\undefined\else
  \let\oldparagraph\paragraph
  \renewcommand{\paragraph}[1]{\oldparagraph{#1}\mbox{}}
\fi
\ifx\subparagraph\undefined\else
  \let\oldsubparagraph\subparagraph
  \renewcommand{\subparagraph}[1]{\oldsubparagraph{#1}\mbox{}}
\fi

\usepackage{color}
\usepackage{fancyvrb}
\newcommand{\VerbBar}{|}
\newcommand{\VERB}{\Verb[commandchars=\\\{\}]}
\DefineVerbatimEnvironment{Highlighting}{Verbatim}{commandchars=\\\{\}}
% Add ',fontsize=\small' for more characters per line
\usepackage{framed}
\definecolor{shadecolor}{RGB}{241,243,245}
\newenvironment{Shaded}{\begin{snugshade}}{\end{snugshade}}
\newcommand{\AlertTok}[1]{\textcolor[rgb]{0.68,0.00,0.00}{#1}}
\newcommand{\AnnotationTok}[1]{\textcolor[rgb]{0.37,0.37,0.37}{#1}}
\newcommand{\AttributeTok}[1]{\textcolor[rgb]{0.40,0.45,0.13}{#1}}
\newcommand{\BaseNTok}[1]{\textcolor[rgb]{0.68,0.00,0.00}{#1}}
\newcommand{\BuiltInTok}[1]{\textcolor[rgb]{0.00,0.23,0.31}{#1}}
\newcommand{\CharTok}[1]{\textcolor[rgb]{0.13,0.47,0.30}{#1}}
\newcommand{\CommentTok}[1]{\textcolor[rgb]{0.37,0.37,0.37}{#1}}
\newcommand{\CommentVarTok}[1]{\textcolor[rgb]{0.37,0.37,0.37}{\textit{#1}}}
\newcommand{\ConstantTok}[1]{\textcolor[rgb]{0.56,0.35,0.01}{#1}}
\newcommand{\ControlFlowTok}[1]{\textcolor[rgb]{0.00,0.23,0.31}{#1}}
\newcommand{\DataTypeTok}[1]{\textcolor[rgb]{0.68,0.00,0.00}{#1}}
\newcommand{\DecValTok}[1]{\textcolor[rgb]{0.68,0.00,0.00}{#1}}
\newcommand{\DocumentationTok}[1]{\textcolor[rgb]{0.37,0.37,0.37}{\textit{#1}}}
\newcommand{\ErrorTok}[1]{\textcolor[rgb]{0.68,0.00,0.00}{#1}}
\newcommand{\ExtensionTok}[1]{\textcolor[rgb]{0.00,0.23,0.31}{#1}}
\newcommand{\FloatTok}[1]{\textcolor[rgb]{0.68,0.00,0.00}{#1}}
\newcommand{\FunctionTok}[1]{\textcolor[rgb]{0.28,0.35,0.67}{#1}}
\newcommand{\ImportTok}[1]{\textcolor[rgb]{0.00,0.46,0.62}{#1}}
\newcommand{\InformationTok}[1]{\textcolor[rgb]{0.37,0.37,0.37}{#1}}
\newcommand{\KeywordTok}[1]{\textcolor[rgb]{0.00,0.23,0.31}{#1}}
\newcommand{\NormalTok}[1]{\textcolor[rgb]{0.00,0.23,0.31}{#1}}
\newcommand{\OperatorTok}[1]{\textcolor[rgb]{0.37,0.37,0.37}{#1}}
\newcommand{\OtherTok}[1]{\textcolor[rgb]{0.00,0.23,0.31}{#1}}
\newcommand{\PreprocessorTok}[1]{\textcolor[rgb]{0.68,0.00,0.00}{#1}}
\newcommand{\RegionMarkerTok}[1]{\textcolor[rgb]{0.00,0.23,0.31}{#1}}
\newcommand{\SpecialCharTok}[1]{\textcolor[rgb]{0.37,0.37,0.37}{#1}}
\newcommand{\SpecialStringTok}[1]{\textcolor[rgb]{0.13,0.47,0.30}{#1}}
\newcommand{\StringTok}[1]{\textcolor[rgb]{0.13,0.47,0.30}{#1}}
\newcommand{\VariableTok}[1]{\textcolor[rgb]{0.07,0.07,0.07}{#1}}
\newcommand{\VerbatimStringTok}[1]{\textcolor[rgb]{0.13,0.47,0.30}{#1}}
\newcommand{\WarningTok}[1]{\textcolor[rgb]{0.37,0.37,0.37}{\textit{#1}}}

\providecommand{\tightlist}{%
  \setlength{\itemsep}{0pt}\setlength{\parskip}{0pt}}\usepackage{longtable,booktabs,array}
\usepackage{calc} % for calculating minipage widths
% Correct order of tables after \paragraph or \subparagraph
\usepackage{etoolbox}
\makeatletter
\patchcmd\longtable{\par}{\if@noskipsec\mbox{}\fi\par}{}{}
\makeatother
% Allow footnotes in longtable head/foot
\IfFileExists{footnotehyper.sty}{\usepackage{footnotehyper}}{\usepackage{footnote}}
\makesavenoteenv{longtable}
\usepackage{graphicx}
\makeatletter
\def\maxwidth{\ifdim\Gin@nat@width>\linewidth\linewidth\else\Gin@nat@width\fi}
\def\maxheight{\ifdim\Gin@nat@height>\textheight\textheight\else\Gin@nat@height\fi}
\makeatother
% Scale images if necessary, so that they will not overflow the page
% margins by default, and it is still possible to overwrite the defaults
% using explicit options in \includegraphics[width, height, ...]{}
\setkeys{Gin}{width=\maxwidth,height=\maxheight,keepaspectratio}
% Set default figure placement to htbp
\makeatletter
\def\fps@figure{htbp}
\makeatother
\newlength{\cslhangindent}
\setlength{\cslhangindent}{1.5em}
\newlength{\csllabelwidth}
\setlength{\csllabelwidth}{3em}
\newlength{\cslentryspacingunit} % times entry-spacing
\setlength{\cslentryspacingunit}{\parskip}
\newenvironment{CSLReferences}[2] % #1 hanging-ident, #2 entry spacing
 {% don't indent paragraphs
  \setlength{\parindent}{0pt}
  % turn on hanging indent if param 1 is 1
  \ifodd #1
  \let\oldpar\par
  \def\par{\hangindent=\cslhangindent\oldpar}
  \fi
  % set entry spacing
  \setlength{\parskip}{#2\cslentryspacingunit}
 }%
 {}
\usepackage{calc}
\newcommand{\CSLBlock}[1]{#1\hfill\break}
\newcommand{\CSLLeftMargin}[1]{\parbox[t]{\csllabelwidth}{#1}}
\newcommand{\CSLRightInline}[1]{\parbox[t]{\linewidth - \csllabelwidth}{#1}\break}
\newcommand{\CSLIndent}[1]{\hspace{\cslhangindent}#1}

\KOMAoption{captions}{tableheading}
\makeatletter
\makeatother
\makeatletter
\@ifpackageloaded{bookmark}{}{\usepackage{bookmark}}
\makeatother
\makeatletter
\@ifpackageloaded{caption}{}{\usepackage{caption}}
\AtBeginDocument{%
\ifdefined\contentsname
  \renewcommand*\contentsname{Table of contents}
\else
  \newcommand\contentsname{Table of contents}
\fi
\ifdefined\listfigurename
  \renewcommand*\listfigurename{List of Figures}
\else
  \newcommand\listfigurename{List of Figures}
\fi
\ifdefined\listtablename
  \renewcommand*\listtablename{List of Tables}
\else
  \newcommand\listtablename{List of Tables}
\fi
\ifdefined\figurename
  \renewcommand*\figurename{Figure}
\else
  \newcommand\figurename{Figure}
\fi
\ifdefined\tablename
  \renewcommand*\tablename{Table}
\else
  \newcommand\tablename{Table}
\fi
}
\@ifpackageloaded{float}{}{\usepackage{float}}
\floatstyle{ruled}
\@ifundefined{c@chapter}{\newfloat{codelisting}{h}{lop}}{\newfloat{codelisting}{h}{lop}[chapter]}
\floatname{codelisting}{Listing}
\newcommand*\listoflistings{\listof{codelisting}{List of Listings}}
\makeatother
\makeatletter
\@ifpackageloaded{caption}{}{\usepackage{caption}}
\@ifpackageloaded{subcaption}{}{\usepackage{subcaption}}
\makeatother
\makeatletter
\@ifpackageloaded{tcolorbox}{}{\usepackage[skins,breakable]{tcolorbox}}
\makeatother
\makeatletter
\@ifundefined{shadecolor}{\definecolor{shadecolor}{rgb}{.97, .97, .97}}
\makeatother
\makeatletter
\makeatother
\makeatletter
\makeatother
\ifLuaTeX
  \usepackage{selnolig}  % disable illegal ligatures
\fi
\IfFileExists{bookmark.sty}{\usepackage{bookmark}}{\usepackage{hyperref}}
\IfFileExists{xurl.sty}{\usepackage{xurl}}{} % add URL line breaks if available
\urlstyle{same} % disable monospaced font for URLs
\hypersetup{
  pdftitle={R\_notes},
  pdfauthor={Wang Anlin},
  colorlinks=true,
  linkcolor={blue},
  filecolor={Maroon},
  citecolor={Blue},
  urlcolor={Blue},
  pdfcreator={LaTeX via pandoc}}

\title{R\_notes}
\author{Wang Anlin}
\date{2023-12-19}

\begin{document}
\maketitle
\ifdefined\Shaded\renewenvironment{Shaded}{\begin{tcolorbox}[sharp corners, interior hidden, borderline west={3pt}{0pt}{shadecolor}, boxrule=0pt, frame hidden, breakable, enhanced]}{\end{tcolorbox}}\fi

\renewcommand*\contentsname{Table of contents}
{
\hypersetup{linkcolor=}
\setcounter{tocdepth}{2}
\tableofcontents
}
\bookmarksetup{startatroot}

\hypertarget{preface}{%
\chapter*{Preface}\label{preface}}
\addcontentsline{toc}{chapter}{Preface}

\markboth{Preface}{Preface}

This is a Quarto book.

To learn more about Quarto books visit
\url{https://quarto.org/docs/books}.

\begin{Shaded}
\begin{Highlighting}[]
\DecValTok{1} \SpecialCharTok{+} \DecValTok{1}
\end{Highlighting}
\end{Shaded}

\begin{verbatim}
[1] 2
\end{verbatim}

\bookmarksetup{startatroot}

\hypertarget{introduction}{%
\chapter{Introduction}\label{introduction}}

This is a book created from markdown and executable code.

See Knuth (1984) for additional discussion of literate programming.

\begin{Shaded}
\begin{Highlighting}[]
\DecValTok{1} \SpecialCharTok{+} \DecValTok{1}
\end{Highlighting}
\end{Shaded}

\begin{verbatim}
[1] 2
\end{verbatim}

\bookmarksetup{startatroot}

\hypertarget{base-r}{%
\chapter*{base R}\label{base-r}}
\addcontentsline{toc}{chapter}{base R}

\markboth{base R}{base R}

\bookmarksetup{startatroot}

\hypertarget{ux6700ux91cdux8981ux7684ux662fux5e2eux52a9ux6587ux6863}{%
\chapter*{最重要的是帮助文档}\label{ux6700ux91cdux8981ux7684ux662fux5e2eux52a9ux6587ux6863}}
\addcontentsline{toc}{chapter}{最重要的是帮助文档}

\markboth{最重要的是帮助文档}{最重要的是帮助文档}

The Comprehensive R Archive Network
\href{https://cran.r-project.org/}{CRAN}

\begin{Shaded}
\begin{Highlighting}[]
\FunctionTok{help.start}\NormalTok{()}
\FunctionTok{help}\NormalTok{()}
\NormalTok{?c}
\end{Highlighting}
\end{Shaded}

\bookmarksetup{startatroot}

\hypertarget{ux5305ux7684ux5b89ux88c5ux548cux4f7fux7528}{%
\chapter*{包的安装和使用}\label{ux5305ux7684ux5b89ux88c5ux548cux4f7fux7528}}
\addcontentsline{toc}{chapter}{包的安装和使用}

\markboth{包的安装和使用}{包的安装和使用}

\begin{Shaded}
\begin{Highlighting}[]
\FunctionTok{installed.packages}\NormalTok{()}
\FunctionTok{install.packages}\NormalTok{(}\StringTok{"ggplot2"}\NormalTok{) }
\FunctionTok{library}\NormalTok{(ggplot2) }\CommentTok{\# 载入包}
\FunctionTok{require}\NormalTok{(ggplot2) }\CommentTok{\# 另一种载入方式}
\FunctionTok{help}\NormalTok{(}\AttributeTok{package =} \StringTok{"ggplot2"}\NormalTok{) }\CommentTok{\# R包ggplot2的帮助文档}
\NormalTok{?ggplot2}
\FunctionTok{data}\NormalTok{(}\AttributeTok{package=}\StringTok{"ggplot2"}\NormalTok{)   }\CommentTok{\#查看R包ggplot2中的数据集}

\FunctionTok{data}\NormalTok{(mpg,}\AttributeTok{package=}\StringTok{"ggplot2"}\NormalTok{)  }\CommentTok{\#加载数据集}
\FunctionTok{help}\NormalTok{(mpg) }\CommentTok{\# 数据集mpg的帮助文档}
\NormalTok{?mpg}
\NormalTok{mpg}
\end{Highlighting}
\end{Shaded}

\bookmarksetup{startatroot}

\hypertarget{ux53d8ux91cfux7c7bux578b}{%
\chapter*{变量类型}\label{ux53d8ux91cfux7c7bux578b}}
\addcontentsline{toc}{chapter}{变量类型}

\markboth{变量类型}{变量类型}

\textbf{数据类型(mode)}:表示对象在计算机内存中的存储类型

\begin{itemize}
\tightlist
\item
  \textbf{numeric -- 数值型(Integer/double)integer -- 整型 L}
\end{itemize}

\begin{Shaded}
\begin{Highlighting}[]
\FunctionTok{mode}\NormalTok{(}\FloatTok{4.3}\NormalTok{)}
\CommentTok{\#\textgreater{} [1] "numeric"}
\FunctionTok{class}\NormalTok{(}\FloatTok{4.3}\NormalTok{)}
\CommentTok{\#\textgreater{} [1] "numeric"}
\FunctionTok{mode}\NormalTok{(1L)}
\CommentTok{\#\textgreater{} [1] "numeric"}
\FunctionTok{class}\NormalTok{(1L)}
\CommentTok{\#\textgreater{} [1] "integer"}
\end{Highlighting}
\end{Shaded}

\begin{itemize}
\tightlist
\item
  \textbf{character -- 字符型}
\end{itemize}

\begin{Shaded}
\begin{Highlighting}[]
\FunctionTok{mode}\NormalTok{(}\FunctionTok{c}\NormalTok{(}\StringTok{"car"}\NormalTok{))}
\CommentTok{\#\textgreater{} [1] "character"}
\FunctionTok{class}\NormalTok{(}\StringTok{"car"}\NormalTok{)}
\CommentTok{\#\textgreater{} [1] "character"}
\end{Highlighting}
\end{Shaded}

\begin{itemize}
\tightlist
\item
  \textbf{logical --逻辑型}
\end{itemize}

\begin{Shaded}
\begin{Highlighting}[]
\FunctionTok{mode}\NormalTok{(}\FunctionTok{c}\NormalTok{(}\ConstantTok{TRUE}\NormalTok{,}\ConstantTok{FALSE}\NormalTok{))}
\CommentTok{\#\textgreater{} [1] "logical"}
\FunctionTok{class}\NormalTok{(}\FunctionTok{c}\NormalTok{(}\ConstantTok{TRUE}\NormalTok{,}\ConstantTok{FALSE}\NormalTok{))}
\CommentTok{\#\textgreater{} [1] "logical"}
\end{Highlighting}
\end{Shaded}

\begin{itemize}
\tightlist
\item
  \textbf{factor--因子}
\end{itemize}

\begin{Shaded}
\begin{Highlighting}[]
\FunctionTok{mode}\NormalTok{(}\FunctionTok{factor}\NormalTok{((}\FunctionTok{c}\NormalTok{(}\DecValTok{1}\NormalTok{,}\DecValTok{2}\NormalTok{,}\DecValTok{3}\NormalTok{)))) }
\CommentTok{\#\textgreater{} [1] "numeric"}
\FunctionTok{class}\NormalTok{(}\FunctionTok{factor}\NormalTok{((}\FunctionTok{c}\NormalTok{(}\DecValTok{1}\NormalTok{,}\DecValTok{2}\NormalTok{,}\DecValTok{3}\NormalTok{)))) }
\CommentTok{\#\textgreater{} [1] "factor"}
\end{Highlighting}
\end{Shaded}

\begin{itemize}
\tightlist
\item
  \textbf{date/datetime--日期/日期时间}
\end{itemize}

默认格式:\texttt{"\%Y-\%m-\%d"\ xxxx-xx-xx,例如:2023-03-15}

\begin{Shaded}
\begin{Highlighting}[]
\FunctionTok{mode}\NormalTok{(}\FunctionTok{as.Date}\NormalTok{(}\StringTok{"2023{-}12{-}11"}\NormalTok{))}
\CommentTok{\#\textgreater{} [1] "numeric"}
\FunctionTok{class}\NormalTok{(}\FunctionTok{as.Date}\NormalTok{(}\StringTok{"2023{-}12{-}11"}\NormalTok{)) }
\CommentTok{\#\textgreater{} [1] "Date"}

\FunctionTok{Sys.Date}\NormalTok{()}
\CommentTok{\#\textgreater{} [1] "2023{-}12{-}19"}
\FunctionTok{as.Date}\NormalTok{(}\FunctionTok{c}\NormalTok{(}\StringTok{"02 14{-}2002"}\NormalTok{,}\StringTok{"01 04{-}2013"}\NormalTok{),}\StringTok{"\%m \%d{-}\%Y"}\NormalTok{) }\CommentTok{\#以"\%m \%d{-}\%Y"格式读入}
\CommentTok{\#\textgreater{} [1] "2002{-}02{-}14" "2013{-}01{-}04"}
\FunctionTok{format}\NormalTok{(}\FunctionTok{Sys.Date}\NormalTok{(),}\StringTok{"\%Y/\%m/\%d"}\NormalTok{) }\CommentTok{\#以"\%Y/\%m/\%d"格式输出}
\CommentTok{\#\textgreater{} [1] "2023/12/19"}
\end{Highlighting}
\end{Shaded}

\begin{itemize}
\tightlist
\item
  \textbf{function--函数}
\end{itemize}

\begin{Shaded}
\begin{Highlighting}[]
\FunctionTok{mode}\NormalTok{(c)}
\CommentTok{\#\textgreater{} [1] "function"}
\FunctionTok{class}\NormalTok{(c)}
\CommentTok{\#\textgreater{} [1] "function"}
\end{Highlighting}
\end{Shaded}

\begin{itemize}
\tightlist
\item
  \textbf{list--列表}
\end{itemize}

\begin{Shaded}
\begin{Highlighting}[]
\FunctionTok{mode}\NormalTok{(mpg)}
\CommentTok{\#\textgreater{} [1] "list"}
\FunctionTok{class}\NormalTok{(mpg)}
\CommentTok{\#\textgreater{} [1] "tbl\_df"     "tbl"        "data.frame"}
\end{Highlighting}
\end{Shaded}

\begin{itemize}
\tightlist
\item
  \textbf{complex -- 复数型}
\end{itemize}

\begin{Shaded}
\begin{Highlighting}[]

\FunctionTok{mode}\NormalTok{(}\FunctionTok{c}\NormalTok{(}\DecValTok{1}\SpecialCharTok{+}\NormalTok{2i,}\DecValTok{3}\SpecialCharTok{{-}}\NormalTok{4i)) }
\CommentTok{\#\textgreater{} [1] "complex"}
\FunctionTok{class}\NormalTok{(}\FunctionTok{c}\NormalTok{(}\DecValTok{1}\SpecialCharTok{+}\NormalTok{2i,}\DecValTok{3}\SpecialCharTok{{-}}\NormalTok{4i))}
\CommentTok{\#\textgreater{} [1] "complex"}
\end{Highlighting}
\end{Shaded}

\begin{itemize}
\tightlist
\item
  \textbf{raw--原始型}
\end{itemize}

\begin{Shaded}
\begin{Highlighting}[]
\FunctionTok{charToRaw}\NormalTok{(}\StringTok{"abcde12345"}\NormalTok{) }\CommentTok{\#字符串中每个字符的原始存储格式(十六进制?)}
\CommentTok{\#\textgreater{}  [1] 61 62 63 64 65 31 32 33 34 35}
\FunctionTok{mode}\NormalTok{(}\FunctionTok{charToRaw}\NormalTok{(}\StringTok{"abcde12345"}\NormalTok{)) }
\CommentTok{\#\textgreater{} [1] "raw"}
\FunctionTok{class}\NormalTok{(}\FunctionTok{charToRaw}\NormalTok{(}\StringTok{"abcde12345"}\NormalTok{))}
\CommentTok{\#\textgreater{} [1] "raw"}
\end{Highlighting}
\end{Shaded}

\bookmarksetup{startatroot}

\hypertarget{ux6570ux636eux7ed3ux6784}{%
\chapter*{数据结构}\label{ux6570ux636eux7ed3ux6784}}
\addcontentsline{toc}{chapter}{数据结构}

\markboth{数据结构}{数据结构}

\textbf{数据结构(class)}:是一种基于\emph{面向对象}的R的抽象类型划分,或者理解为一种数据结构。

\begin{itemize}
\tightlist
\item
  \textbf{vector --向量},同一类元素的集合。
\end{itemize}

\texttt{?c\ \ \ \ \ \ \#Combine\ Values\ into\ a\ Vector\ or\ List}

\begin{Shaded}
\begin{Highlighting}[]
\CommentTok{\#单元素向量}
\DecValTok{1}
\CommentTok{\#\textgreater{} [1] 1}

\FunctionTok{is.vector}\NormalTok{(}\DecValTok{1}\NormalTok{)}
\CommentTok{\#\textgreater{} [1] TRUE}

\StringTok{"a"}
\CommentTok{\#\textgreater{} [1] "a"}
\FunctionTok{is.vector}\NormalTok{(}\StringTok{"a"}\NormalTok{)}
\CommentTok{\#\textgreater{} [1] TRUE}

\CommentTok{\#使用函数c()创建向量}

\FunctionTok{c}\NormalTok{(}\DecValTok{1}\NormalTok{)}
\CommentTok{\#\textgreater{} [1] 1}
\FunctionTok{c}\NormalTok{(}\DecValTok{1}\NormalTok{,}\DecValTok{2}\NormalTok{,}\DecValTok{3}\NormalTok{,}\DecValTok{4}\NormalTok{,}\DecValTok{5}\NormalTok{)}
\CommentTok{\#\textgreater{} [1] 1 2 3 4 5}
\FunctionTok{c}\NormalTok{(}\StringTok{"a"}\NormalTok{,}\StringTok{"b"}\NormalTok{,}\StringTok{"c"}\NormalTok{)}
\CommentTok{\#\textgreater{} [1] "a" "b" "c"}
\end{Highlighting}
\end{Shaded}

\begin{itemize}
\tightlist
\item
  \textbf{factor --因子},分类变量,其中每个级别都是一类。
\end{itemize}

\texttt{?factor}

\texttt{factor(vector,order=FALSE,levels=c(v1,v2,…),labels=\ ,...)}
,在内存中以整数\texttt{c(1,2,3,...,k)}形式存储。

\begin{Shaded}
\begin{Highlighting}[]
\CommentTok{\#类别(名义型)变量}
\NormalTok{diabetes}\OtherTok{\textless{}{-}}\FunctionTok{c}\NormalTok{(}\StringTok{"t1"}\NormalTok{,}\StringTok{"t2"}\NormalTok{,}\StringTok{"t1"}\NormalTok{,}\StringTok{"t1"}\NormalTok{) }
\FunctionTok{str}\NormalTok{(diabetes)}
\CommentTok{\#\textgreater{}  chr [1:4] "t1" "t2" "t1" "t1"}
\NormalTok{diabetes}\OtherTok{\textless{}{-}}\FunctionTok{factor}\NormalTok{(diabetes)}
\FunctionTok{str}\NormalTok{(diabetes)}
\CommentTok{\#\textgreater{}  Factor w/ 2 levels "t1","t2": 1 2 1 1}


\CommentTok{\#有序型变量     默认水平根据字母顺序而定}
\NormalTok{status}\OtherTok{\textless{}{-}}\FunctionTok{c}\NormalTok{(}\StringTok{"poor"}\NormalTok{,}\StringTok{"better"}\NormalTok{,}\StringTok{"best"}\NormalTok{,}\StringTok{"poor"}\NormalTok{)}
\NormalTok{status}\OtherTok{\textless{}{-}}\FunctionTok{factor}\NormalTok{(status,}\AttributeTok{order=}\ConstantTok{TRUE}\NormalTok{) }
\FunctionTok{str}\NormalTok{(status) }
\CommentTok{\#\textgreater{}  Ord.factor w/ 3 levels "best"\textless{}"better"\textless{}..: 3 2 1 3}
\NormalTok{status}\OtherTok{\textless{}{-}}\FunctionTok{factor}\NormalTok{(status,}\AttributeTok{order=}\ConstantTok{TRUE}\NormalTok{,}\AttributeTok{levels =} \FunctionTok{c}\NormalTok{(}\StringTok{"poor"}\NormalTok{,}\StringTok{"better"}\NormalTok{,}\StringTok{"best"}\NormalTok{)) }
\FunctionTok{str}\NormalTok{(status) }
\CommentTok{\#\textgreater{}  Ord.factor w/ 3 levels "poor"\textless{}"better"\textless{}..: 1 2 3 1}

\CommentTok{\#改变外在标签}
\NormalTok{sex}\OtherTok{\textless{}{-}}\FunctionTok{c}\NormalTok{(}\DecValTok{1}\NormalTok{,}\DecValTok{2}\NormalTok{,}\DecValTok{2}\NormalTok{,}\DecValTok{1}\NormalTok{)}
\NormalTok{sex}
\CommentTok{\#\textgreater{} [1] 1 2 2 1}
\NormalTok{sex}\OtherTok{\textless{}{-}}\FunctionTok{factor}\NormalTok{(sex,}\AttributeTok{levels=}\FunctionTok{c}\NormalTok{(}\DecValTok{1}\NormalTok{,}\DecValTok{2}\NormalTok{),}\AttributeTok{labels =} \FunctionTok{c}\NormalTok{(}\StringTok{"男"}\NormalTok{,}\StringTok{"女"}\NormalTok{)) }
\FunctionTok{str}\NormalTok{(sex) }
\CommentTok{\#\textgreater{}  Factor w/ 2 levels "男","女": 1 2 2 1}
\NormalTok{sex}
\CommentTok{\#\textgreater{} [1] 男 女 女 男}
\CommentTok{\#\textgreater{} Levels: 男 女}
\end{Highlighting}
\end{Shaded}

\begin{itemize}
\tightlist
\item
  \textbf{matrix --矩阵},所有元素必须是同一类型。
\end{itemize}

\texttt{?matrix}

\texttt{matrix(data=\ ,nrow=1\ ,ncol=1\ ,byrow=FALSE\ ,dimnames=list(rnames,cnames)\ ,...)}

\begin{Shaded}
\begin{Highlighting}[]
\NormalTok{num}\OtherTok{\textless{}{-}}\FunctionTok{c}\NormalTok{(}\DecValTok{16}\NormalTok{,}\DecValTok{22}\NormalTok{,}\DecValTok{24}\NormalTok{,}\DecValTok{28}\NormalTok{)}
\NormalTok{rnames}\OtherTok{\textless{}{-}}\FunctionTok{c}\NormalTok{(}\StringTok{"R1"}\NormalTok{,}\StringTok{"R2"}\NormalTok{)}
\NormalTok{cnames}\OtherTok{\textless{}{-}}\FunctionTok{c}\NormalTok{(}\StringTok{"C1"}\NormalTok{,}\StringTok{"C2"}\NormalTok{)}
\NormalTok{mymatrix}\OtherTok{\textless{}{-}}\FunctionTok{matrix}\NormalTok{(num,}\AttributeTok{nrow=}\DecValTok{2}\NormalTok{,}\AttributeTok{ncol=}\DecValTok{2}\NormalTok{,}\AttributeTok{byrow=}\ConstantTok{TRUE}\NormalTok{,}\AttributeTok{dimnames=}\FunctionTok{list}\NormalTok{(rnames,cnames))}
\NormalTok{mymatrix   }
\CommentTok{\#\textgreater{}    C1 C2}
\CommentTok{\#\textgreater{} R1 16 22}
\CommentTok{\#\textgreater{} R2 24 28}
\end{Highlighting}
\end{Shaded}

\begin{itemize}
\tightlist
\item
  \textbf{array--数组},所有元素必须是同一类型。
\end{itemize}

\texttt{?array}

\texttt{array(data,dim\_numeric\_vector,dimnames\ =\ list(dim1,dim2,...),...)}

\begin{Shaded}
\begin{Highlighting}[]
\NormalTok{v}\OtherTok{\textless{}{-}}\DecValTok{1}\SpecialCharTok{:}\DecValTok{24} 
\NormalTok{dim1}\OtherTok{\textless{}{-}}\FunctionTok{c}\NormalTok{(}\StringTok{"A1"}\NormalTok{,}\StringTok{"A2"}\NormalTok{,}\StringTok{"A3"}\NormalTok{) }
\NormalTok{dim2}\OtherTok{\textless{}{-}}\FunctionTok{c}\NormalTok{(}\StringTok{"B1"}\NormalTok{,}\StringTok{"B2"}\NormalTok{,}\StringTok{"B3"}\NormalTok{,}\StringTok{"B4"}\NormalTok{)}
\NormalTok{dim3}\OtherTok{\textless{}{-}}\FunctionTok{c}\NormalTok{(}\StringTok{"C1"}\NormalTok{,}\StringTok{"C2"}\NormalTok{) }
\NormalTok{myarray}\OtherTok{\textless{}{-}}\FunctionTok{array}\NormalTok{(v,}\FunctionTok{c}\NormalTok{(}\DecValTok{3}\NormalTok{,}\DecValTok{4}\NormalTok{,}\DecValTok{2}\NormalTok{),}\AttributeTok{dimnames =} \FunctionTok{list}\NormalTok{(dim1,dim2,dim3)) }
\NormalTok{myarray}
\CommentTok{\#\textgreater{} , , C1}
\CommentTok{\#\textgreater{} }
\CommentTok{\#\textgreater{}    B1 B2 B3 B4}
\CommentTok{\#\textgreater{} A1  1  4  7 10}
\CommentTok{\#\textgreater{} A2  2  5  8 11}
\CommentTok{\#\textgreater{} A3  3  6  9 12}
\CommentTok{\#\textgreater{} }
\CommentTok{\#\textgreater{} , , C2}
\CommentTok{\#\textgreater{} }
\CommentTok{\#\textgreater{}    B1 B2 B3 B4}
\CommentTok{\#\textgreater{} A1 13 16 19 22}
\CommentTok{\#\textgreater{} A2 14 17 20 23}
\CommentTok{\#\textgreater{} A3 15 18 21 24}
\end{Highlighting}
\end{Shaded}

\begin{itemize}
\tightlist
\item
  \textbf{data.frame/tibble-- 数据框},
  由列向量组成,每一列元素必须是同一类型,列与列之间长度必须相同,但类型可以不同。
\end{itemize}

\texttt{?data.frame}

\texttt{data.frame(name1=col1,name2=col2,...,row.names\ =\ ,...)}

\texttt{?tibble},\texttt{tibble()}是tidyverse风格的数据框,用法类似。

\begin{Shaded}
\begin{Highlighting}[]
\NormalTok{id}\OtherTok{\textless{}{-}}\FunctionTok{c}\NormalTok{(}\DecValTok{1}\NormalTok{,}\DecValTok{2}\NormalTok{,}\DecValTok{3}\NormalTok{,}\DecValTok{4}\NormalTok{) }
\NormalTok{age}\OtherTok{\textless{}{-}}\FunctionTok{c}\NormalTok{(}\DecValTok{21}\NormalTok{,}\DecValTok{14}\NormalTok{,}\DecValTok{52}\NormalTok{,}\DecValTok{15}\NormalTok{) }
\NormalTok{diabetes}\OtherTok{\textless{}{-}}\FunctionTok{c}\NormalTok{(}\StringTok{"t1"}\NormalTok{,}\StringTok{"t2"}\NormalTok{,}\StringTok{"t1"}\NormalTok{,}\StringTok{"t1"}\NormalTok{) }
\NormalTok{status}\OtherTok{\textless{}{-}}\FunctionTok{c}\NormalTok{(}\StringTok{"poor"}\NormalTok{,}\StringTok{"better"}\NormalTok{,}\StringTok{"best"}\NormalTok{,}\StringTok{"poor"}\NormalTok{) }
\NormalTok{patient}\OtherTok{\textless{}{-}}\FunctionTok{data.frame}\NormalTok{(}\AttributeTok{patientID=}\NormalTok{id,age,diabetes,status,}\AttributeTok{row.names =}\NormalTok{ id) }\CommentTok{\# 4个列向量组成数据框 }
\NormalTok{patient}
\CommentTok{\#\textgreater{}   patientID age diabetes status}
\CommentTok{\#\textgreater{} 1         1  21       t1   poor}
\CommentTok{\#\textgreater{} 2         2  14       t2 better}
\CommentTok{\#\textgreater{} 3         3  52       t1   best}
\CommentTok{\#\textgreater{} 4         4  15       t1   poor}
\end{Highlighting}
\end{Shaded}

list --列表,由以上各种数据结构组成

\texttt{?list}

\texttt{list(name1=object1,name2=object2,...)}

\begin{Shaded}
\begin{Highlighting}[]
\NormalTok{mylist}\OtherTok{\textless{}{-}}\FunctionTok{list}\NormalTok{(}\AttributeTok{title=}\StringTok{"My list"}\NormalTok{,}
             \AttributeTok{matr=}\FunctionTok{matrix}\NormalTok{(}\FunctionTok{c}\NormalTok{(}\StringTok{"a1"}\NormalTok{,}\StringTok{"b1"}\NormalTok{,}\StringTok{"a2"}\NormalTok{,}\StringTok{"b2"}\NormalTok{),}\AttributeTok{nrow=}\DecValTok{2}\NormalTok{,}\AttributeTok{ncol=}\DecValTok{2}\NormalTok{,}\AttributeTok{byrow=}\ConstantTok{TRUE}\NormalTok{,}
                         \AttributeTok{dimnames =} \FunctionTok{list}\NormalTok{(}\FunctionTok{c}\NormalTok{(}\StringTok{"X1"}\NormalTok{,}\StringTok{"X2"}\NormalTok{),}\FunctionTok{c}\NormalTok{(}\StringTok{"Y1"}\NormalTok{,}\StringTok{"Y2"}\NormalTok{))}
\NormalTok{             ),}
             \AttributeTok{df=}\FunctionTok{data.frame}\NormalTok{(}\AttributeTok{id=}\FunctionTok{matrix}\NormalTok{(}\FunctionTok{c}\NormalTok{(}\StringTok{"Lisa"}\NormalTok{,}\StringTok{"BOb"}\NormalTok{,}\StringTok{"John"}\NormalTok{,}\StringTok{"Jule"}\NormalTok{),}
                                     \AttributeTok{nrow=}\DecValTok{4}\NormalTok{,}\AttributeTok{ncol=}\DecValTok{1}\NormalTok{,}\AttributeTok{byrow=}\ConstantTok{TRUE}
\NormalTok{             ),}
             \AttributeTok{int=}\FunctionTok{c}\NormalTok{(}\DecValTok{3}\NormalTok{,}\DecValTok{5}\NormalTok{,}\DecValTok{7}\NormalTok{,}\DecValTok{9}\NormalTok{),}
             \AttributeTok{TF=}\FunctionTok{c}\NormalTok{(T,T,T,F)}
\NormalTok{             ),}
             \AttributeTok{list=}\FunctionTok{list}\NormalTok{(}\AttributeTok{a=}\FunctionTok{c}\NormalTok{(}\DecValTok{1}\NormalTok{,}\DecValTok{2}\NormalTok{,}\DecValTok{3}\NormalTok{),}\AttributeTok{b=}\FunctionTok{c}\NormalTok{(}\StringTok{"A"}\NormalTok{,}\StringTok{"B"}\NormalTok{))}
\NormalTok{)}
\NormalTok{mylist}
\CommentTok{\#\textgreater{} $title}
\CommentTok{\#\textgreater{} [1] "My list"}
\CommentTok{\#\textgreater{} }
\CommentTok{\#\textgreater{} $matr}
\CommentTok{\#\textgreater{}    Y1   Y2  }
\CommentTok{\#\textgreater{} X1 "a1" "b1"}
\CommentTok{\#\textgreater{} X2 "a2" "b2"}
\CommentTok{\#\textgreater{} }
\CommentTok{\#\textgreater{} $df}
\CommentTok{\#\textgreater{}     id int    TF}
\CommentTok{\#\textgreater{} 1 Lisa   3  TRUE}
\CommentTok{\#\textgreater{} 2  BOb   5  TRUE}
\CommentTok{\#\textgreater{} 3 John   7  TRUE}
\CommentTok{\#\textgreater{} 4 Jule   9 FALSE}
\CommentTok{\#\textgreater{} }
\CommentTok{\#\textgreater{} $list}
\CommentTok{\#\textgreater{} $list$a}
\CommentTok{\#\textgreater{} [1] 1 2 3}
\CommentTok{\#\textgreater{} }
\CommentTok{\#\textgreater{} $list$b}
\CommentTok{\#\textgreater{} [1] "A" "B"}
\end{Highlighting}
\end{Shaded}

\bookmarksetup{startatroot}

\hypertarget{qmd_format}{%
\chapter*{qmd\_format}\label{qmd_format}}
\addcontentsline{toc}{chapter}{qmd\_format}

\markboth{qmd\_format}{qmd\_format}

\begin{longtable}[]{@{}
  >{\raggedright\arraybackslash}p{(\columnwidth - 2\tabcolsep) * \real{0.5128}}
  >{\raggedright\arraybackslash}p{(\columnwidth - 2\tabcolsep) * \real{0.4872}}@{}}
\toprule\noalign{}
\endhead
\bottomrule\noalign{}
\endlastfoot
输出 & 语法 \\
\texttt{code} & `\texttt{code}` \\
\textbf{粗体} & \texttt{**粗体**} \\
\emph{斜体} & \texttt{*斜体*} \\
\ul{下横线} & \texttt{{[}下横线{]}\{.underline\}} \\
\st{删除线} &
\texttt{\textasciitilde{}\textasciitilde{}删除线\textasciitilde{}\textasciitilde{}} \\
上标X\textsuperscript{2} & \texttt{X\^{}2\^{}} \\
下标 X\textsubscript{1} &
\texttt{X\textasciitilde{}1\textasciitilde{}} \\
\textsc{SAMALL small caps} & 与小写字母同等高度的大写字母
\texttt{{[}small\ caps{]}\{.smallcaps\}} \\
\url{https://r4ds.hadley.nz/} &
\texttt{\textless{}https://r4ds.hadley.nz/\textgreater{}} \\
\href{https://quarto.org}{quarto} &
\texttt{{[}quarto{]}(https://quarto.org)} \\
\includegraphics{images/whole-game.png} &
\texttt{!{[}data\ science{]}(images/whole-game.png)} \\
\end{longtable}

\bookmarksetup{startatroot}

\hypertarget{summary}{%
\chapter{Summary}\label{summary}}

In summary, this book has no content whatsoever.

\begin{Shaded}
\begin{Highlighting}[]
\DecValTok{1} \SpecialCharTok{+} \DecValTok{1}
\end{Highlighting}
\end{Shaded}

\begin{verbatim}
[1] 2
\end{verbatim}

\bookmarksetup{startatroot}

\hypertarget{references}{%
\chapter*{References}\label{references}}
\addcontentsline{toc}{chapter}{References}

\markboth{References}{References}

\hypertarget{refs}{}
\begin{CSLReferences}{1}{0}
\leavevmode\vadjust pre{\hypertarget{ref-knuth84}{}}%
Knuth, Donald E. 1984. {``Literate Programming.''} \emph{Comput. J.} 27
(2): 97--111. \url{https://doi.org/10.1093/comjnl/27.2.97}.

\end{CSLReferences}



\end{document}
